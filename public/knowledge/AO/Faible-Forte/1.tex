\documentclass{article}
\usepackage{amsmath}
\usepackage{amsfonts}

\title{Unification de la Force Faible et de la Force Forte via le Théorème All is One}
\author{Votre Nom}
\date{}

\begin{document}

\maketitle

\begin{abstract}
Dans le cadre du \textit{Théorème All is One (AO)}, la force faible et la force forte, bien qu'apparaissant comme des forces distinctes à basse énergie, sont des résonances vibratoires dans un champ unifié. À des énergies extrêmement élevées, ces deux forces se synchronisent et se manifestent comme une seule interaction unifiée. La séparation de ces forces à basse énergie est due à une désynchronisation des fréquences vibratoires.
\end{abstract}

\section{1. Forces Faible et Forte comme Résonances Vibratoires}

La force faible et la force forte peuvent être comprises comme des fréquences vibratoires distinctes dans un champ unifié :
\begin{itemize}
    \item \textbf{Force Faible} : Représente une vibration de haute fréquence, responsable des interactions subatomiques à courte portée.
    \item \textbf{Force Forte} : Représente une vibration intense, mais confinée à une portée encore plus courte, responsable de la cohésion des quarks et des noyaux atomiques.
\end{itemize}

\section{2. Unification à Haute Énergie}

À des niveaux d'énergie extrêmement élevés, les fréquences vibratoires de la force faible et de la force forte se rejoignent en résonance parfaite, formant une seule force unifiée. Cette résonance est représentée par l'équation :

\[
\Psi(\mathcal{F}, t) = A_w e^{i\omega_w t} + A_s e^{i\omega_s t}
\]

Où :
\begin{itemize}
    \item \( A_w \) est l'amplitude de la force faible,
    \item \( A_s \) est l'amplitude de la force forte,
    \item \( \omega_w \) est la fréquence de la force faible,
    \item \( \omega_s \) est la fréquence de la force forte,
    \item \( t \) est le temps.
\end{itemize}

\section{3. Séparation à Basse Énergie}

À basse énergie, les fréquences vibratoires des forces faible et forte se désynchronisent, entraînant une séparation apparente des deux forces :
\begin{itemize}
    \item La \textbf{force faible} continue de se manifester à l'échelle subatomique avec une fréquence élevée.
    \item La \textbf{force forte} agit avec une intensité plus grande mais confinée à une portée encore plus courte.
\end{itemize}

\section{4. Conclusion}

Le \textit{Théorème All is One} propose que la force faible et la force forte sont des résonances vibratoires différenciées dans un champ unifié. À haute énergie, ces forces se synchronisent pour former une interaction unifiée, tandis qu'à basse énergie, elles se désynchronisent, apparaissant comme des forces distinctes.

\end{document}
