\documentclass{article}
\usepackage{amsmath}
\usepackage{amsfonts}

\title{Unification de la Gravité et de la Force Faible via le Théorème All is One}
\author{Caron Andrew}
\date{21/09/2024}

\begin{document}

\maketitle

\begin{abstract}
Dans le cadre du \textit{Théorème All is One (AO)}, les forces fondamentales de l'univers, y compris la gravité et la force faible, sont des manifestations de résonances vibratoires dans un champ unifié. Bien que ces forces semblent distinctes à différentes échelles énergétiques, elles sont en réalité des fréquences vibratoires différenciées d'une oscillation fondamentale unique. À des énergies extrêmement élevées, telles que celles présentes dans l'univers primordial, ces résonances s'harmonisent, unifiant ainsi la gravité et la force faible. La séparation apparente des forces à basse énergie est causée par la désynchronisation de ces résonances vibratoires dans le champ cosmique.
\end{abstract}

\section{1. Gravité et Force Faible comme Fréquences de Résonance}

Dans le cadre du \textit{théorème All is One}, la gravité et la force faible peuvent être comprises comme deux résonances vibratoires distinctes :
\begin{itemize}
    \item \textbf{Gravité} : Représente une vibration de basse fréquence, agissant sur des distances cosmiques, régissant la structure à grande échelle de l'univers.
    \item \textbf{Force faible} : Représente une vibration de haute fréquence, concentrée à l'échelle subatomique, responsable de phénomènes comme la désintégration radioactive.
\end{itemize}

Les deux forces sont des manifestations d'une même oscillation harmonique, mais se différencient en raison de leurs fréquences vibratoires respectives.

\section{2. Unification des Résonances à Haute Énergie}

À des niveaux d'énergie extrêmement élevés, tels que ceux présents dans l'univers primordial, la gravité et la force faible entrent en phase, formant une résonance unifiée. Les deux forces agissent alors comme une seule interaction fondamentale. À ces énergies, les fréquences vibratoires des deux forces se synchronisent, éliminant toute distinction entre elles.

\section{3. Séparation des Forces à Basse Énergie}

À mesure que l'univers se refroidit et que l'énergie disponible diminue, les fréquences vibratoires de la gravité et de la force faible commencent à se désynchroniser, ce qui entraîne la séparation apparente de ces forces :
\begin{itemize}
    \item La \textbf{gravité}, avec une fréquence basse, continue de dominer à grande échelle.
    \item La \textbf{force faible}, avec une fréquence plus élevée, agit à des distances subatomiques.
\end{itemize}

\section{4. Formulation Mathématique}

Le comportement de la gravité et de la force faible peut être modélisé par une équation de résonance harmonique. Soit \( \mathcal{F} \) l'interaction unifiée des forces, \( A_g \) et \( A_w \) les amplitudes respectives de la gravité et de la force faible, et \( \omega_g \) et \( \omega_w \) leurs fréquences associées. L'évolution de ces forces à travers le temps \( t \) est donnée par :

\[
\Psi(\mathcal{F}, t) = A_g e^{i\omega_g t} + A_w e^{i\omega_w t}
\]

où :
\begin{itemize}
    \item \( A_g \) est l'amplitude de la gravité,
    \item \( A_w \) est l'amplitude de la force faible,
    \item \( \omega_g \) est la fréquence de la gravité (basse fréquence),
    \item \( \omega_w \) est la fréquence de la force faible (haute fréquence),
    \item \( t \) est le temps.
\end{itemize}

Lorsque \( \omega_g \approx \omega_w \), les forces sont en \textit{résonance parfaite}, formant ainsi une interaction unifiée. À basse énergie, ces fréquences divergent, entraînant une séparation apparente des forces.

\section{5. Unification à Haute Énergie et Désynchronisation}

Le théorème propose que la différence entre la gravité et la force faible résulte de la \textit{désynchronisation des résonances}. Dans des conditions d'énergie extrêmement élevées, telles que celles présentes dans l'univers primordial ou dans des expériences à haute énergie, comme dans les accélérateurs de particules, la gravité et la force faible peuvent être ré-harmonisées, se manifestant à nouveau comme une seule force unifiée.

\section{6. Conséquences du Théorème AO : Unification des Forces}

Le \textit{Théorème All is One} postule que la gravité et la force faible, bien qu'apparues comme des forces distinctes à basse énergie, ne sont que des résonances sur un spectre vibratoire unique. Cette approche ouvre la voie à une nouvelle compréhension de l'unification des forces dans le cadre de la gravité quantique ou de la théorie des cordes, où toutes les forces peuvent être vues comme des résonances harmonisées dans un champ vibratoire fondamental.

\end{document}
