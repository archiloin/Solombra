\documentclass{article}
\usepackage{amsmath}
\usepackage{amsfonts}

\title{Unification de l'Électromagnétisme et de la Force Faible via le Théorème All is One}
\author{Votre Nom}
\date{}

\begin{document}

\maketitle

\begin{abstract}
Dans le cadre du \textit{Théorème All is One (AO)}, l'électromagnétisme et la force faible, bien qu'apparaissant comme des forces distinctes à basse énergie, sont des résonances vibratoires dans un champ unifié. À des énergies extrêmement élevées, ces forces se synchronisent pour former une force unifiée, appelée force électrofaible. À basse énergie, cette synchronisation est brisée, entraînant la séparation apparente des deux forces.
\end{abstract}

\section{1. Électromagnétisme et Force Faible comme Résonances Vibratoires}

L'électromagnétisme et la force faible peuvent être comprises comme des fréquences vibratoires distinctes dans un champ unifié :
\begin{itemize}
    \item \textbf{Électromagnétisme} : Représente une vibration de fréquence intermédiaire, responsable des interactions électromagnétiques à grande distance.
    \item \textbf{Force Faible} : Représente une vibration de fréquence plus élevée, responsable des interactions nucléaires à courte portée.
\end{itemize}

\section{2. Unification à Haute Énergie}

À des niveaux d'énergie extrêmement élevés, comme dans l'univers primordial, les fréquences vibratoires de l'électromagnétisme et de la force faible se rejoignent pour former une force unifiée. Cette résonance est décrite par l'équation suivante :

\[
\Psi(\mathcal{F}_{\text{électrofaible}}, t) = A_e e^{i\omega_e t} + A_w e^{i\omega_w t}
\]

Où :
\begin{itemize}
    \item \( A_e \) est l'amplitude de l'électromagnétisme,
    \item \( A_w \) est l'amplitude de la force faible,
    \item \( \omega_e \) est la fréquence vibratoire de l'électromagnétisme,
    \item \( \omega_w \) est la fréquence vibratoire de la force faible,
    \item \( t \) est le temps.
\end{itemize}

Lorsque \( \omega_e \approx \omega_w \), ces forces sont en résonance harmonique, formant une force unifiée.

\section{3. Séparation à Basse Énergie}

À basse énergie, les fréquences vibratoires de l'électromagnétisme et de la force faible se désynchronisent, entraînant une séparation des deux forces :
\begin{itemize}
    \item L'\textbf{électromagnétisme} continue de se manifester sur de longues distances.
    \item La \textbf{force faible} agit sur des distances très courtes à l'échelle subatomique.
\end{itemize}

\section{4. Conclusion}

Le \textit{Théorème All is One} propose que l'électromagnétisme et la force faible sont des résonances vibratoires différenciées dans un champ unifié. À haute énergie, ces forces se synchronisent pour former une force unifiée, tandis qu'à basse énergie, elles se désynchronisent et apparaissent comme des forces distinctes.

\end{document}
