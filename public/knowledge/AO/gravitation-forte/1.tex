\documentclass{article}
\usepackage{amsmath}
\usepackage{amsfonts}

\title{Unification de la Gravité et de la Force Forte via le Théorème All is One}
\author{Caron Andrew}
\date{21/09/2024}

\begin{document}

\maketitle

\begin{abstract}
Dans le cadre du \textit{Théorème All is One (AO)}, les forces fondamentales de l'univers, y compris la gravité et la force forte, sont des manifestations de résonances vibratoires dans un champ unifié. Ces forces, bien qu'elles apparaissent distinctes à différentes échelles, sont en réalité des fréquences vibratoires de la même oscillation fondamentale. À des niveaux d'énergie extrêmement élevés, ces résonances s'harmonisent, unifiant ainsi la gravité et la force forte en une seule interaction fondamentale. La séparation de ces forces à basse énergie est le résultat de la désynchronisation des résonances vibratoires dans le champ cosmique.
\end{abstract}

\section{1. Gravité et Force Forte comme Fréquences de Résonance}

Dans ce cadre, la gravité, une force faible agissant à grande échelle, et la force forte, une force extrêmement puissante mais limitée à l'échelle subatomique, sont perçues comme deux résonances vibratoires distinctes :
\begin{itemize}
    \item \textbf{Gravité} : Représente une vibration de basse fréquence qui agit à grande échelle cosmique, régissant la structure de l'univers.
    \item \textbf{Force forte} : Représente une vibration de haute fréquence, concentrée à l'échelle subatomique, maintenant la cohésion des particules élémentaires.
\end{itemize}

Les deux forces sont des manifestations d'une même oscillation harmonique qui se différencie en raison de leurs fréquences respectives.

\section{2. Unification des Résonances à Haute Énergie}

À des niveaux d'énergie extrêmement élevés, comme dans l'univers primitif juste après le Big Bang, la gravité et la force forte étaient en phase, créant une résonance unifiée. Les forces agissaient alors comme une \textit{unique force fondamentale}. À ces énergies, les résonances étaient harmonisées, et la distinction entre gravité et force forte n'existait pas encore.

\section{3. Séparation des Forces à Basse Énergie}

À mesure que l'univers s'est refroidi et que l'énergie a diminué, les résonances des forces se sont désynchronisées, donnant lieu à la séparation de la gravité et de la force forte en deux interactions distinctes :
\begin{itemize}
    \item La \textbf{gravité} est devenue une résonance à basse fréquence, influençant des échelles macroscopiques.
    \item La \textbf{force forte} a conservé sa résonance de haute fréquence, mais restreinte à l'échelle subatomique.
\end{itemize}

\section{4. Formulation Mathématique}

L'interaction entre la gravité et la force forte peut être modélisée par une équation de résonance harmonique. Soit \( \mathcal{F} \) l'interaction unifiée des forces, \( A_g \) et \( A_s \) les amplitudes respectives de la gravité et de la force forte, et \( \omega_g \) et \( \omega_s \) leurs fréquences associées. L'évolution de ces forces à travers le temps \( t \) est donnée par :

\[
\Psi(\mathcal{F}, t) = A_g e^{i\omega_g t} + A_s e^{i\omega_s t}
\]

où :
\begin{itemize}
    \item \( A_g \) est l'amplitude de la gravité,
    \item \( A_s \) est l'amplitude de la force forte,
    \item \( \omega_g \) est la fréquence de la gravité (basse fréquence),
    \item \( \omega_s \) est la fréquence de la force forte (haute fréquence),
    \item \( t \) est le temps.
\end{itemize}

Lorsque \( \omega_g \approx \omega_s \), la gravité et la force forte sont en \textit{résonance parfaite}, et les deux forces se manifestent comme une interaction unifiée. À basse énergie, \( \omega_g \) et \( \omega_s \) se désynchronisent, entraînant la séparation des forces.

\section{5. Unification à Haute Énergie et Désynchronisation Cosmique}

Le théorème propose que les différences observées entre la gravité et la force forte sont dues à des \textit{désynchronisations résonantes}. En présence d'énergies suffisamment élevées, comme celles produites dans les accélérateurs de particules ou dans les premiers instants de l'univers, la gravité et la force forte peuvent être \textit{ré-harmonisées} et se manifester à nouveau comme une seule force unifiée.

\section{6. Conséquences du Théorème AO : Unification des Forces}

Le \textit{Théorème All is One} suggère que l'unification des forces dans un champ vibratoire est réalisable, offrant une nouvelle perspective sur la recherche d'une théorie unifiée des forces fondamentales. Cette approche propose que la gravité et la force forte, bien que distinctes à basse énergie, ne sont que des fréquences sur un spectre commun et peuvent être ré-harmonisées à haute énergie.

\end{document}
