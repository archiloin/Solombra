\documentclass{article}
\usepackage{amsmath}
\usepackage{amsfonts}

\title{Unification de l'Électromagnétisme et de la Force Forte via le Théorème All is One}
\author{Caron Andrew}
\date{21/09/2024}

\begin{document}

\maketitle

\begin{abstract}
Dans le cadre du \textit{Théorème All is One (AO)}, l'électromagnétisme et la force forte, bien qu'apparaissant comme des forces distinctes à basse énergie, sont des résonances vibratoires dans un champ unifié. À des énergies extrêmement élevées, ces deux forces se synchronisent pour former une force unifiée. La séparation de ces forces à basse énergie est due à la désynchronisation des fréquences vibratoires associées à chacune.
\end{abstract}

\section{1. Électromagnétisme et Force Forte comme Résonances Vibratoires}

L'électromagnétisme et la force forte peuvent être comprises comme des fréquences vibratoires distinctes dans un champ unifié :
\begin{itemize}
    \item \textbf{Électromagnétisme} : Représente une vibration de fréquence intermédiaire, responsable des interactions entre les particules chargées à longue portée.
    \item \textbf{Force Forte} : Représente une vibration de fréquence élevée, responsable des interactions entre quarks à l'intérieur des nucléons.
\end{itemize}

\section{2. Unification à Haute Énergie}

À des niveaux d'énergie extrêmement élevés, les fréquences vibratoires de l'électromagnétisme et de la force forte se synchronisent pour former une force unifiée. Cette résonance harmonique est décrite par l'équation suivante :

\[
\Psi(\mathcal{F}, t) = A_e e^{i\omega_e t} + A_s e^{i\omega_s t}
\]

Où :
\begin{itemize}
    \item \( A_e \) est l'amplitude de l'électromagnétisme,
    \item \( A_s \) est l'amplitude de la force forte,
    \item \( \omega_e \) est la fréquence vibratoire de l'électromagnétisme,
    \item \( \omega_s \) est la fréquence vibratoire de la force forte,
    \item \( t \) est le temps.
\end{itemize}

Lorsque \( \omega_e \approx \omega_s \), ces forces sont en résonance harmonique, formant une force unifiée.

\section{3. Séparation à Basse Énergie}

À basse énergie, les fréquences vibratoires de l'électromagnétisme et de la force forte se désynchronisent, entraînant une séparation des deux forces :
\begin{itemize}
    \item L'\textbf{électromagnétisme} agit à longue portée sur les particules chargées.
    \item La \textbf{force forte} agit à très courte portée sur les quarks à l'intérieur des nucléons.
\end{itemize}

\section{4. Conclusion}

Le \textit{Théorème All is One} propose que l'électromagnétisme et la force forte sont des résonances vibratoires différenciées dans un champ unifié. À haute énergie, ces forces se synchronisent pour former une force unifiée, tandis qu'à basse énergie, elles se désynchronisent et apparaissent comme des forces distinctes.

\end{document}
