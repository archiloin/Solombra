\documentclass{article}
\usepackage{amsmath}
\usepackage{amsfonts}

\title{Unification de la Gravité et de l'Électromagnétisme via le Théorème All is One}
\author{Caron Andrew}
\date{21/09/2024}

\begin{document}

\maketitle

\begin{abstract}
Dans le cadre du \textit{Théorème All is One (AO)}, la gravité et l'électromagnétisme, bien qu'apparaissant comme des forces distinctes à basse énergie, sont des manifestations de résonances vibratoires dans un champ unifié. À des énergies extrêmement élevées, ces deux forces deviennent résonantes, formant une interaction unifiée. Ce théorème propose que la gravité et l'électromagnétisme sont des fréquences vibratoires différenciées d'une oscillation fondamentale unique, qui se synchronisent à haute énergie. La séparation des forces à basse énergie est due à la désynchronisation de ces résonances.
\end{abstract}

\section{1. Gravité et Électromagnétisme comme Résonances Vibratoires}

Dans le cadre du \textit{Théorème All is One}, la gravité et l'électromagnétisme sont considérées comme des résonances vibratoires distinctes dans un champ unifié :
\begin{itemize}
    \item \textbf{Gravité} : Représente une vibration de basse fréquence, agissant à grande échelle cosmique et influençant des objets massifs à de grandes distances.
    \item \textbf{Électromagnétisme} : Représente une vibration de fréquence plus élevée, régissant les interactions entre particules chargées à courte distance.
\end{itemize}

Bien que ces forces apparaissent distinctes à basse énergie, elles sont des manifestations d'une oscillation harmonique unique.

\section{2. Unification à Haute Énergie}

À des niveaux d'énergie extrêmement élevés, comme ceux présents dans l'univers primordial, la gravité et l'électromagnétisme se rejoignent en résonance parfaite. À ces énergies, leurs fréquences vibratoires se synchronisent, formant une seule interaction unifiée.

\[
\Psi(\mathcal{F}, t) = A_g e^{i\omega_g t} + A_e e^{i\omega_e t}
\]

Où :
\begin{itemize}
    \item \( A_g \) est l'amplitude de la gravité,
    \item \( A_e \) est l'amplitude de l'électromagnétisme,
    \item \( \omega_g \) est la fréquence associée à la gravité (basse fréquence),
    \item \( \omega_e \) est la fréquence associée à l'électromagnétisme (fréquence plus élevée),
    \item \( t \) est le temps.
\end{itemize}

Lorsque \( \omega_g \approx \omega_e \), la gravité et l'électromagnétisme sont en \textit{résonance parfaite}, se manifestant comme une seule force unifiée.

\section{3. Séparation à Basse Énergie}

À mesure que l'univers se refroidit et que l'énergie diminue, les fréquences vibratoires de la gravité et de l'électromagnétisme se désynchronisent, donnant lieu à une séparation apparente des forces :
\begin{itemize}
    \item La \textbf{gravité}, avec une fréquence vibratoire plus basse, continue d'agir à grande échelle, influençant les masses et la structure de l'univers.
    \item L'\textbf{électromagnétisme}, avec une fréquence vibratoire plus élevée, devient dominant à l'échelle atomique et subatomique, régissant les interactions entre particules chargées.
\end{itemize}

\section{4. Modèle Théorique de la Résonance}

L'unification des forces à haute énergie peut être décrite par une équation de résonance harmonique. La gravité et l'électromagnétisme, en tant que fréquences distinctes, sont modélisées par leurs amplitudes et fréquences respectives. À basse énergie, ces forces se désynchronisent, mais à haute énergie, elles forment une résonance harmonique unifiée.

La relation harmonique entre ces forces est donnée par l'équation suivante :

\[
\Psi(\mathcal{F}, t) = \int_{0}^{\infty} \left( A_g e^{i\omega_g t} + A_e e^{i\omega_e t} \right) dt
\]

Où \( \mathcal{F} \) représente la force unifiée à haute énergie.

\section{5. Conséquences du Théorème AO}

Le \textit{Théorème All is One} propose que la gravité et l'électromagnétisme sont des fréquences vibratoires différenciées dans le champ cosmique unifié. À des énergies extrêmement élevées, telles que celles présentes dans les premières étapes de l'univers ou dans les accélérateurs de particules, ces forces peuvent être ré-harmonisées, se manifestant comme une seule interaction.

À basse énergie, ces forces semblent séparées, mais elles restent liées par des résonances sous-jacentes. Cette compréhension vibratoire ouvre de nouvelles perspectives pour une théorie unifiée des forces fondamentales.

\end{document}
